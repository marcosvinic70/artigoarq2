\documentclass[12pt]{article}

\usepackage{sbc-template}

\usepackage{graphicx,url}

\usepackage[brazil]{babel}   
%\usepackage[latin1]{inputenc}  
\usepackage[utf8]{inputenc}  
% UTF-8 encoding is recommended by ShareLaTex

     
\sloppy

\title{Micro-arquitetura ZEN e sua implementação no Ryzen 7}

\author{Roberto Gonçalves Santos Júnior\inst{1}, Rafael H. Bordini\inst{2}, Flávio Rech
  Wagner\inst{1}, Jomi F. Hübner\inst{3} }


\address{Departamento de Ciência da Computação -- Universidade Federal de Lavras
  (UFLA)\\
  Caixa Postal 15.064 -- 91.501-970 -- Lavras -- MG -- Brazil
\nextinstitute
  Departamento de Ciência da Computação\\
  Universidade Federal de Lavras -- Lavras, MG -- Brazil
  \email{\{roberto.goncalves\}@computacao.ufla.br, R.Bordini@durham.ac.uk,
  jomi@inf.furb.br}
}

\begin{document} 

\maketitle

\begin{abstract}
  This paper describes a technical analysis about the ZEN microarchitecture provided by AMD. This analysis consists on the each block of ZEN architecture and their detailed description about how it works. Futhermore, discuss what has been implemented on Ryzen 7.
\end{abstract}
     
\begin{resumo} 
  Esse artigo descreve a analise técnica sobre a microarquitetura ZEN desenvolvida pela AMD. Essa analise consiste na separação de cada bloco da arquitetura ZEN e sua descrição detalhada de como ela funciona. Além disso, discutir o que foi implementado no Ryzen 7. 
\end{resumo}



\section{A arquitetura ZEN no Ryzen 7}

A arquitetura ZEN é muito diferente das antigas arquiteturas da AMD, e segundo a AMD, com ela, eles conseguiram alcançar "40\% IPC(Instructions per clock) speedup em relação a arquitetura do  Excavator" o que nessa seção, será discutido no Ryzen 7 1700.\\

\begin{figure}[ht]
\centering
\includegraphics[width=120mm,scale=0.8]{AMD-energy-per-cycle.jpg}
\caption{Comparação ZEN, fonte: AMD}
\label{fig:AMD CORE}
\end{figure}

É importante salientar que segundo o gráfico, houve 40\% ganho de IPC e o mesmo número de energia por ciclo do Excavator, esse é um dos motivos da grande surpresa na arquitetura ZEN.


\begin{figure}[ht]
\centering
\includegraphics[width=120mm,scale=0.8]{AMD-Zen_CPU-Complex.png}
\caption{CPU Complex, fonte: AMD}
\label{fig:AMD CORE}
\end{figure}

A CPU Complex conhecida como CCX é o conjunto de núcleos da CPU. É importante ressaltar que cada CCX está conectada a uma cache L3. Sendo que, a associatividade da Cache L3 é 16-way, e que ela é feita de 4 pedaços, por low-order. E por fim, cada núcleo pode acessar cada cache como a mesma latência média.

Essa mudança foi crucial para o desempenho do Zen sobre o Excavator, na relação em que aumentou-se o número de pipelines de 4 para 6 na cache L1 e houve uma duplicação da quantidade de bits para a FMAC(fused multiply–add capability). Além disso, acrescentou-se a MMX Unit(Matrix Math Extensions), ou seja, uma unidade auxiliar para efetuar envolvendo matrizes. 

O aumento da duplicação da quantidade de bits para a FMAC, possibilita o aumento da quantidade de instruções que podem ser executadas em cada FMAC. Outra modificação importante é uma instruction decoder para Inteiro e FP, que no Excavator era divida. A unidade de decodificação é responsável, basicamente, por determinar qual instrução será executada depois que passou pelo intruction memory.


\begin{figure}[ht]
\centering
\includegraphics[width=115mm,scale=0.8]{zenex.jpg}
\caption{CPU Complex, fonte: AMD}
\label{fig:AMD CORE}
\end{figure}

A partir dessas informações, com o aumento da capacidade das FMACs, investimento na Cache L3 de cada CCX e o aumento do número de pipelines da cache L1, pode ocorrer que para alguns softwares o IPC do ZEN supere o IPC do Excavator.

\section{General Information}

All full papers and posters (short papers) submitted to some SBC conference,
including any supporting documents, should be written in English or in
Portuguese. The format paper should be A4 with single column, 3.5 cm for upper
margin, 2.5 cm for bottom margin and 3.0 cm for lateral margins, without
headers or footers. The main font must be Times, 12 point nominal size, with 6
points of space before each paragraph. Page numbers must be suppressed.

Full papers must respect the page limits defined by the conference.
Conferences that publish just abstracts ask for \textbf{one}-page texts.

\section{First Page} \label{sec:firstpage}

The first page must display the paper title, the name and address of the
authors, the abstract in English and ``resumo'' in Portuguese (``resumos'' are
required only for papers written in Portuguese). The title must be centered
over the whole page, in 16 point boldface font and with 12 points of space
before itself. Author names must be centered in 12 point font, bold, all of
them disposed in the same line, separated by commas and with 12 points of
space after the title. Addresses must be centered in 12 point font, also with
12 points of space after the authors' names. E-mail addresses should be
written using font Courier New, 10 point nominal size, with 6 points of space
before and 6 points of space after.

The abstract and ``resumo'' (if is the case) must be in 12 point Times font,
indented 0.8cm on both sides. The word \textbf{Abstract} and \textbf{Resumo},
should be written in boldface and must precede the text.

\section{CD-ROMs and Printed Proceedings}

In some conferences, the papers are published on CD-ROM while only the
abstract is published in the printed Proceedings. In this case, authors are
invited to prepare two final versions of the paper. One, complete, to be
published on the CD and the other, containing only the first page, with
abstract and ``resumo'' (for papers in Portuguese).

\section{Sections and Paragraphs}

Section titles must be in boldface, 13pt, flush left. There should be an extra
12 pt of space before each title. Section numbering is optional. The first
paragraph of each section should not be indented, while the first lines of
subsequent paragraphs should be indented by 1.27 cm.

\subsection{Subsections}

The subsection titles must be in boldface, 12pt, flush left.

\section{Figures and Captions}\label{sec:figs}


Figure and table captions should be centered if less than one line
(Figure~\ref{fig:exampleFig1}), otherwise justified and indented by 0.8cm on
both margins, as shown in Figure~\ref{fig:exampleFig2}. The caption font must
be Helvetica, 10 point, boldface, with 6 points of space before and after each
caption.

\begin{figure}[ht]
\centering
\includegraphics[width=.5\textwidth]{fig1.jpg}
\caption{A typical figure}
\label{fig:exampleFig1}
\end{figure}

\begin{figure}[ht]
\centering
\includegraphics[width=.3\textwidth]{fig2.jpg}
\caption{This figure is an example of a figure caption taking more than one
  line and justified considering margins mentioned in Section~\ref{sec:figs}.}
\label{fig:exampleFig2}
\end{figure}

In tables, try to avoid the use of colored or shaded backgrounds, and avoid
thick, doubled, or unnecessary framing lines. When reporting empirical data,
do not use more decimal digits than warranted by their precision and
reproducibility. Table caption must be placed before the table (see Table 1)
and the font used must also be Helvetica, 10 point, boldface, with 6 points of
space before and after each caption.

\begin{table}[ht]
\centering
\caption{Variables to be considered on the evaluation of interaction
  techniques}
\label{tab:exTable1}
\smallskip
\begin{tabular}{|l|c|c|}
\hline
& Value 1 & Value 2\\[0.5ex]
\hline
&&\\[-2ex]
Case 1 & 1.0 $\pm$ 0.1 & 1.75$\times$10$^{-5}$ $\pm$ 5$\times$10$^{-7}$\\[0.5ex]
\hline
&&\\[-2ex]
Case 2 & 0.003(1) & 100.0\\[0.5ex]
\hline
\end{tabular}
\end{table}

\section{Images}

All images and illustrations should be in black-and-white, or gray tones,
excepting for the papers that will be electronically available (on CD-ROMs,
internet, etc.). The image resolution on paper should be about 600 dpi for
black-and-white images, and 150-300 dpi for grayscale images.  Do not include
images with excessive resolution, as they may take hours to print, without any
visible difference in the result. 

\section{References}

Bibliographic references must be unambiguous and uniform.  We recommend giving
the author names references in brackets, e.g. \cite{knuth:84},
\cite{boulic:91}, and \cite{smith:99}.

The references must be listed using 12 point font size, with 6 points of space
before each reference. The first line of each reference should not be
indented, while the subsequent should be indented by 0.5 cm.

\bibliographystyle{sbc}
\bibliography{sbc-template}

\end{document}
